%%%%%%%%%%%%%%%%%%%
% IMPORTANT!
% The following code must not be changed
%%%%%%%%%%%%%%%%%%%

\documentclass[12pt,a4paper]{article}
\usepackage{geometry}
\usepackage{graphics,graphicx}
\usepackage{amsmath,amssymb}
%\usepackage[portuges]{babel}
\usepackage[utf8]{inputenc}
\geometry{a4paper,tmargin=25mm,bmargin=20mm,lmargin=25mm,rmargin=25mm}


%%%%%%%%%%%%%%%%%%%
%
% Dear seminar coordinator, please complete the data below
% 
%%%%%%%%%%%%%%%%%%%

\newcommand{\area}{Teoria da Computação}
\newcommand{\hora}{ 10:00 }
\newcommand{\local}{Sala: MAT B - térreo do MAT}
\newcommand{\dia}{25/05/18}

%%%%%%%%%%%%%%%%%%%
%
% Dear speaker, please complete the data below
% 
%%%%%%%%%%%%%%%%%%%

\newcommand{\titulo}{Functional Nominal C-Unification}
\newcommand{\palestrante}{Gabriel Silva}
\newcommand{\universidade}{Universidade de Brasília}


\begin{document}








\pagestyle{empty}

\bigskip

\begin{center}
{ \Large  {\bf\sc Semin\'ario de \area}}
\end{center}



\bigskip
\bigskip

\begin{center}

{\Large\bf \titulo}

\vspace{1cm}

{\large\bf \palestrante}

{\universidade}

\vspace{1.0cm}

\dia \\
\hora  Horas\vspace{0.2cm}

\local
\end{center}

\vspace{0.8cm}


%%%%%%%%%%%%%%%%%%%%%%%%%%%%%%%%%%%%%%%%%%

\noindent\textbf{Abstract}. 
Nominal syntax extends first-order syntax bringing mechanisms to deal 
with bound and free variables, that is with $\alpha$-conversion, in a natural manner. 
Profiting from the nominal paradigm implies adapting basic notions such as
substitution, rewriting, equality to it. Recent work presents advances in nominal
unification modulo equational theories. In this
talk the problem of nominal unification with commutative operators is revisited, and
a functional algorithm for nominal C-unification is presented. This is a joint work
with Mauricio Ayala-Rincón.


%%%%%%%%%%%%%%%%%%%%%%%%%%%%%%%%%%%%%%%%%%%%%%%%%


\begin{thebibliography}{999}


\bibitem{bib1}

Fernández, Maribel, and Murdoch J. Gabbay. "Nominal rewriting." \textit{Information and
Computation} 205.6 (2007): 917-965.

\bibitem{bib2} 

Ayala-Rincón, Mauricio, et al. "Nominal C-Unification." \textit{arXiv preprint
arXiv}:1709.05384 (2017).

\end{thebibliography}

\end{document}

