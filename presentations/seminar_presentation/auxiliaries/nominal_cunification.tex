\section{Nominal C-Unification}
\subsection{Definition of the Problem}
% what is a unification problem - TODO: check how this is done
\begin{frame}{Unification Problem}
    \begin{definition}[Unification Problem] 
        A unification problem is a pair $\langle \Delta, P \rangle$, where $\Delta$ is 
        a freshness context and $P$ is a finite set of equations ($s \eqUnif t$) and freshness
        constraints ($a \freshPr s$).
    \end{definition}
\end{frame}

% solution for a unification problem 
\begin{frame}{Solution to a Unification Problem} 
    \begin{definition}[Solution to a Unification Problem]
        The pair $\langle \nabla, \sigma \rangle$ is a 
        solution for a triple $\mathcal{P} = \langle \Delta, \delta, P \rangle$ when
        \begin{itemize}
            \item $\nabla \vdash \Delta \sigma$
            \item $\nabla \vdash a\#t\sigma$, if $a\freshPr t \in P$
            \item $\nabla \vdash s\sigma \aeq t\sigma$, if $s \aeq t \in P$ 
            \item There exist $\lambda$ such that $\Delta \vdash \delta\lambda \aeq
                \sigma$
        \end{itemize}
    \end{definition}
    The unification problem $\langle \Delta, P \rangle$ is associated with the triple 
    $\langle \Delta, id, P \rangle$.
\end{frame}

% more general solutions
% TODO: if there is time, explain what is a complete set of solutions
% or i can take this slide down
%\begin{frame}{More General Solutions}
%    \begin{definition}[More General Solution]
%        $\langle \Delta, \sigma \rangle$ is more general than $\langle \Delta',
%        \sigma' \rangle$ when 
%        \begin{itemize}
%            \item There exist $\lambda$ satisfying $\Delta' \vdash \sigma\lambda \aeq
%                \sigma'$
%            \item $\Delta' \vdash \Delta \lambda$
%        \end{itemize}
%    \end{definition}
%\end{frame}

% distictions from nominal unification: fixpoint equations and set of solutions
\subsection{Differences from Nominal Syntactic Unification}
\begin{frame}{Difference from Syntactic Unification - Fixpoint Equations}
     \par A fixpoint equation is of the form $ \pi \cdot X \aeq \pi' \cdot X$. In syntactic nominal
     unification, this is resolved by adding to the context the constraints that
     guarantee that 
     atoms that are affected in different ways by $\pi$ and $\pi'$ must be fresh in
     X.  
\end{frame}
\begin{frame}{Difference from Syntactic Unification - Fixpoint Equations}
     \par However, this approach is not complete in C-unification, because of
     commutativity. 
     \par Take for example $\pi = (a \ b)$, $\pi' = nil$ and the
     substitution $X \rightarrow a+b$. In this case, $b + a \aeq a+b$. However,
     the treatment of syntactic unification would lose this solution, since neither
     $a$ or $b$ are fresh in $a+b$. 
     \par Because of that, fixpoint equations are part of the solution for a
     C-unification problem. 
\end{frame}

\begin{frame}{Difference from Syntactic Unification - Set of Solutions}
    \par Let $f$ be a commutative symbol. The equation $f(t_0, t_1) \aeq f(s_0, s_1)$
    can be solved by solving ($t_0 \aeq s_0$ and $t_1 \aeq s_1$) OR by solving 
    ($t_0 \aeq s_1$ and $t_1 \aeq s_0$). 
    \newline
    \par This means we need to consider two branches. Since each branch can generate
    solutions, there is now a set of solutions to be obtained, instead of only one.  
\end{frame}


% my algorithm - this is in another file now. This is old version, don't use it. 
%\begin{frame}[allowframebreaks]{A Functional Nominal C-Unification Algorithm}
%\begin{algorithmic}[1]
%\Procedure{unify}{$\Delta,\sigma,UnPrb, FxPntEq$}
%    \If{null($UnPrb$)} 
%        \State \Return {list(($\Delta, \sigma, FxPntEq$))} 
%    \Else 
%    \State $t =$ head($UnPrb$)[1]
%    \State $s =$ head($UnPrb$)[2]
%    \State $UnPrb' =$ tail($UnPrb$)
%        \If {($s == \pi \cdot X$) and ($X$ not in $t$)}
%            \State $\sigma' = \{X \rightarrow t \}$
%            \State $\sigma''$ = $\sigma' \cup \sigma$ 
%            \State ($\Delta'$, bool1) = appSub2Ctxt($\sigma_1, \Delta$) 
%            \State $UnPrb' = (UnPrb)\sigma' +   (FxPntEq)\sigma'$
%            % quick and dirty fix since i could not handle nested ifs. 
%            % if then else
%            \Statex
%            \State \textbf{if} bool1 \textbf{then} \Return{\Call{unify}{$\Delta',
%            \sigma'', UnPrb', null$}} 
%            \State \textbf{else} \Return{null}
%        \Else 
%            \If{$t == a$} 
%                % if
%                \State \textbf{if} s == a \textbf{then} 
%                \State \hspace{8 \algorithmicxindent} 
%                \Return{\Call{unify}{$\Delta, \sigma, UnPrb', FxPntEq$}} 
%                % else
%                \State \textbf{else} \Return{null}
%            \ElsIf{$t == \pi \cdot X$}
%                % if 
%                \State \textbf{if} (X not in s) \textbf{then}
%                \State \hspace{8 \algorithmicxindent}
%                \Comment Similar to case above where s is a suspension
%                % else if
%                \Statex \Statex
%                \State \textbf{else if} $(s == \pi' \cdot X)$ \textbf{then}
%                \State \hspace{8 \algorithmicxindent}
%                $FxPntEq' = FxPntEq \cup \{((\pi')^{-1} \oplus \pi) \cdot X\}$
%                \State \hspace{8 \algorithmicxindent}
%                \Return{\Call{unify}{$\Delta, \sigma, UnPrb, FxPntEq'$}}
%                % else 
%                \State \textbf{else} \Return{null}
%
%            % unit
%            \ElsIf{$t == \langle \rangle$}
%                \State \textbf{if} $s == \langle \rangle$ \textbf{then} 
%                \State \hspace{8 \algorithmicxindent}
%                \Return{\Call{unify}{$\Delta, \sigma, UnPrb, FxPntEq$}}
%                \State \textbf{else} \Return{null}
%            % pair
%            \ElsIf{$t == \langle t_1, t_2 \rangle$}
%                \State \textbf{if} $s == \langle s_1, s_2 \rangle$ \textbf{then} 
%                \State \hspace{8 \algorithmicxindent}
%                $UnPrb'' = UnPrb' + [(s_1, t_1)] + [(s_2, t_2)]$
%                \State \hspace{8 \algorithmicxindent}
%                \Return{\Call{unify}{$\Delta, \sigma, UnPrb'', FxPntEq$}}
%                \State \textbf{else} \Return{null}
%            % abs
%            \ElsIf{$t == [a]t_1$}
%                \State \textbf{if} $s == [a]s_1$ \textbf{then}        
%                    \State \hspace{8 \algorithmicxindent}
%                    $UnPrb'' = UnPrb' + [(t_1, s_1)] + [(t_2, s_2)]$
%                    \State \hspace{8 \algorithmicxindent}
%                    \Return{\Call{unify}{$\Delta, \sigma, UnPrb'', FxPntEq$}}
%                \State \textbf{else if} $s == [b]s_1$ \textbf{then}        
%                    \State \hspace{8 \algorithmicxindent}
%                    $(\Delta', bool1) =  fresh(a, s_1)$ 
%                    \State \hspace{8 \algorithmicxindent}
%                    $\Delta'' = \Delta \cup \Delta'$ 
%                    \State \hspace{8 \algorithmicxindent}
%                    $UnPrb'' = UnPrb + [(t_1, \ \ (a \ b) \ s_1)]$ 
%                    \State \hspace{8 \algorithmicxindent}
%                    \textbf{if} bool1 \textbf{then} 
%                    \State \hspace{16 \algorithmicxindent}
%                    \Return{\Call{unify}{$\Delta'', \sigma, UnPrb'', FxPntEq$}}
%                    \State \hspace{8 \algorithmicxindent}
%                    \textbf{else} \Return{null}
%                \State \textbf{else} 
%                    \State \hspace{8 \algorithmicxindent}
%                    \Return{null}
%            \ElsIf{$t == f t_1$} \Comment f is not commutative
%                \State \textbf{if} $s$ != $f s_1$ \textbf{then} \Return{null}
%                \State \textbf{else} 
%                    \State \hspace{8 \algorithmicxindent}
%                    $UnPrb'' = UnPrb + [(t_1, s_1)]$ 
%                    \State \hspace{8 \algorithmicxindent}
%                    \Return{\Call{unify}{$\Delta, \sigma, UnPrb'', FxPntEq$}}
%            \Statex\Statex\Statex\Statex
%            \Statex\Statex\Statex\Statex
%            \Else \Comment $t$ is of the form $f(t_1, t_2)$
%                %\State \textbf{if} $s$ != $f(s_1, s_2)$ \textbf{then} \Return{null}
%                \If{$s$ != $f(s_1, s_2)$} \Return{null}
%                \Else
%                    \State $UnPrb_1 = UnPrb' + [(s_1, t_1)] + [(s_2, t_2)]$
%                    \State $s_1$ = \Return{\Call{unify}{$\Delta, \sigma, UnPrb_1, FxPntEq$}}
%                    \State $UnPrb_2 = UnPrb' + [(s_1, t_2)] + [(s_2, t_1)]$
%                    \State $s_2$ = \Return{\Call{unify}{$\Delta, \sigma, UnPrb_2, FxPntEq$}}
%                    \State \Return{\Call{append}{$s_1, s_2$}}
%                \Endif
%            \Endif
%        \Endif
%    \Endif
%\EndProcedure
%\end{algorithmic}
%\end{frame}

% remaining parts of algorithm 
% TODO: do this slide, if there is time!
%\begin{frame}{Reduction Rules for Equational Problems}
%    TO DO 
%\end{frame}
%
%\begin{frame}{Reduction Rules for Freshness Problems}
%    TO DO 
%\end{frame}

% do one example on the blackboard
