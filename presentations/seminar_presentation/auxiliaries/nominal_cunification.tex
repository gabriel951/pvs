\section{Nominal C Unification}
\subsection{Definition of the Problem}
% what is a unification problem - TODO: check how this is done
\begin{frame}{Unification Problem}
    \begin{definition}[Unification Problem] 
        A unification problem is a pair $\langle \Delta, P \rangle$, where $\Delta$ is 
        a freshness context and $P$ is a finite set of equations ($s \eqUnif t$) and freshness
        constraints ($a \freshPr s$).
    \end{definition}
\end{frame}

% solution for a unification problem 
\begin{frame}{Solution to a Unification Problem} 
    \begin{definition}[Solution to a Unification Problem]
        TO DO.
        %A solution for a triple \mathbb{P} = 
    \end{definition}
\end{frame}

% more general solutions
\begin{frame}{More General Solutions}
    \begin{definition}[More General Solution]
        TO DO 
    \end{definition}
\end{frame}

% the algorithm
\subsection{A Nominal C Unification Algorithm}

\begin{frame}
    Hi
\end{frame}

% general idea for my algorithm 
\begin{frame}[allowframebreaks]{An Algorithm for Nominal C-Unification: General Idea}
    \begin{algorithmic}[1]
        \Procedure{unify}{$\Delta,\sigma,UnPrb, FxPntEq$}
        \If{null($UnPrb$)} 
            \State \textbf{return} list(($\Delta, \sigma, FxPntEq$)) 
        \Else 
        \State $t =$ head($UnPrb$)[1]
        \State $s =$ head($UnPrb$)[2]
            \If {($s == \pi \cdot X$) and ($X$ not in $t$)}
                \State $\sigma_1 = \{X \rightarrow t \}$
                \State $\sigma'$ = $\sigma_1 \cup \sigma$ 
                \State ($\Delta'$, bool1) = appSub2Ctxt($\sigma_1, \Delta$) 
                \State $UnPrb'$ = $UnPrb\sigma_1$ \cup $FxPntEq\sigma_1$

            \ElsIf{huuuuuHIdjklasf}
                \State Hi
            \ElsIf{HI}
                \State Hi
            \ElsIf{HI}
                \State Hi
            \Else 
                \State Hey
            \Endif
        \Endif
        \EndProcedure
    \end{algorithmic}
\end{frame}

% remaining parts of algorithm 
\begin{frame}{Reduction Rules for Equational Problems}
    TO DO 
\end{frame}

\begin{frame}{Reduction Rules for Freshness Problems}
    TO DO 
\end{frame}

% distictions from nominal unification: fixpoint equations and set of solutions
\begin{frame}{Difference from Nominal Unification - Fixpoint Equations}
    TO DO
\end{frame}

\begin{frame}{Difference from Nominal Unification - Set of Solutions}
    TO DO
\end{frame}

\subsection{Examples}
% copy paste washington examples here
\begin{frame}{Example 1}
    TO DO
\end{frame}

\begin{frame}{Example 2}
    TO DO 
\end{frame}
