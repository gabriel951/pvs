\section{Background}
\subsection{Nominal Terms, Permutations and Substitutions}
\begin{frame}{Atoms and Variables} 
    Consider a set of variables $\varSet = \{X, Y, Z, \dots \}$ and 
    a set of atoms $\atomSet = \{a, b, c, \dots \}$.
\end{frame}

% permutation
\begin{frame}{Permutations}
    An atom permutation $\pi$ represents an exchange of a finite amount of atoms in
    $\atomSet$
    and is represented by a list of swappings: 
\begin{equation*}
    \pi = (a_1 \ b_1):: ... ::(a_n \ b_n)::nil 
\end{equation*}
\end{frame}

% definition of nominal terms
\begin{frame}{Nominal Terms}
    \begin{definition}[Nominal Terms]
    Nominal terms are inductively generated according to the grammar: 
    \begin{equation*}
        s,t \ \ ::= \ \ \langle \rangle \ \ | \ \ a \ \ | \ \ \pi \cdot X \ \ | \ \ [a]t \ \ |
                    \ \ \langle s, t \rangle \ \       | \ \ f \ t \ \   
    \end{equation*}
    \end{definition}
    The symbols denote respectively: unit, atom term, suspended variable,
    abstraction, pair and function application. 
    \newline
    We impose a restriction on the syntax
    of commutative function symbols: they must receive pairs. 
\end{frame}

% permutation action definition
%\begin{frame}{Permutation Action on an Atom}
%    \begin{definition}[Permutation Action on an Atom]
%    The action of a permutation on an atom is defined by induction: 
%    \begin{itemize} 
%        \item $nil \cdot a \coloneqq a$ 
%        \item $(b \ c):: \pi \cdot a \coloneqq \pi \cdot a$ 
%        \item $(b \ c):: \pi \cdot b \coloneqq \pi \cdot c$ 
%    \end{itemize}
%    \end{definition}
%\end{frame}

%\begin{frame}{Permutation Action on a Term}
%    \begin{definition}[Permutation Action on a Term]
%    The action of a permutation $\pi$ on a term $t$ is defined by induction:
%    \begin{align*}
%        \pi \cdot  \langle \rangle &= \langle \rangle & 
%        \pi \cdot (\pi' \cdot X) &= (\pi \oplus \pi') \cdot X & 
%        \\ 
%        \pi \cdot \langle s, t \rangle &= \langle \pi \cdot s, \pi \cdot t \rangle  & 
%        \pi \cdot ([a]t) &= [\pi \cdot a](\pi \cdot t) & 
%        \\
%        \pi \cdot (f \ t) &= f(\pi \cdot t) 
%    \end{align*}
%    \end{definition}
%\end{frame}

% Examples of Permutation Action
\begin{frame}{Examples of Permutation Actions}
    Permutations act on atoms and terms:
    \begin{itemize}
        \item $t = a$, $\pi = (a \ b)$, $\pi \cdot t = b$.
        \item $t = f(a, c)$, $\pi = (a \ b)$ and $\pi \cdot t = f(b, c)$.
        \item $t = [a]a$, $\pi = (a \ b)::(b \ c)$, $\pi \cdot t = [c]c$.
    \end{itemize}
\end{frame}

% substitution
\begin{frame}{Substitution}
    \begin{definition}[Substitution]
    A substitution $\sigma$ is a mapping from variables to terms, such that
    $\{X \ | \ X \neq X\sigma \}$ is finite.
    \end{definition}
\end{frame}

% substitution action
%\begin{frame}{Substitution Action on Terms}
%    \begin{definition}[Substitution Action on a Term]    
%        The action of a substitution $\sigma$ on a term $t$ is denoted by $\sigma t$ and
%        defined recursively by: 
%    \begin{align*}
%        \langle \rangle \sigma &= \langle \rangle & 
%        (\pi \cdot X) \sigma &= \pi \cdot (X \sigma)  & 
%        \\ 
%        \langle s, t \rangle \sigma &= \langle  s \sigma, t \sigma \rangle  & 
%        ([a]t) \sigma &= [a](t \sigma) & 
%        \\
%        (f \ t) \sigma &= f (t \sigma) 
%    \end{align*}
%    
%    OBS: Substitutions and permutations commute. 
%    \end{definition}
%\end{frame}

% TODO
\begin{frame}{Examples of Substitutions Acting on Terms}
    Substitutions also act on terms: 
    \begin{itemize}
        \item $\sigma = \{X \rightarrow a\}$, $t = f(X, X)$, $t\sigma = f(a, a)$.
        \item $\sigma = \{X \rightarrow f(a, b)\}$, $t = f(X, c)$, $t\sigma = f(f(a, b), c)$.
    \end{itemize}
\end{frame}

\subsection{Freshness and $\alpha$-Equality}

\begin{frame}{Intuition Behind the Concepts}
    Two important predicates are the freshness predicate $\#$ and the $\alpha$-equality
    predicate $\aeq$: 
    \begin{itemize}
        \item $a\#t$ means that if $a$ occurs in $t$ then it must do so
            under an abstractor $[a]$. 
        \item $s \aeq t$ means that $s$ and $t$ are $\alpha$-equivalent.
    \end{itemize}
\end{frame}

\begin{frame}{Contexts}
    A context is a set of constraints of the form $a\#X$. Contexts are denoted by the letters
    $\Delta$, $\nabla$ or $\Gamma$.
\end{frame}

\begin{frame}{Derivation Rules for Freshness}
    \begin{tabular}{ c c }
    % fresh empty tuple
        \AxiomC{}
        \RightLabel{($\# \langle \rangle$)}
        \UnaryInfC{$\Delta \vdash a \# \langle \rangle$}
        \DisplayProof
    &
    % fresh atom
        \AxiomC{}
        \RightLabel{($\# atom$)}
        \UnaryInfC{$\Delta \vdash a \# b$}
        \DisplayProof
    \\ \\  
    % perm - fresh
        \AxiomC{$(\pi^{-1}(a) \# X) \in \Delta $}
        \RightLabel{($\# X$)}
        \UnaryInfC{$\Delta \vdash a \# \pi \cdot X$}
        \DisplayProof
    &
    % abstraction - a 
        \AxiomC{}
        \RightLabel{($\# [a]a$)}
        \UnaryInfC{$\Delta \vdash a \# [a]t$}
        \DisplayProof
    \\ \\ 
    % abstraction - b 
        \AxiomC{$\Delta \vdash a \# t$}
        \RightLabel{($\# [a]b$)}
        \UnaryInfC{$\Delta \vdash a \# [b]t$}
        \DisplayProof
    & 
    % pair
        \AxiomC{$\Delta \vdash a \# s \ \ \ \Delta \vdash a \# t$}
        \RightLabel{($\# pair$)}
        \UnaryInfC{$\Delta \vdash a \# \langle s, t \rangle$}
        \DisplayProof
    \\ \\
    % application of function
        \AxiomC{$\Delta \vdash a \# t $}
        \RightLabel{($\# app$)}
        \UnaryInfC{$\Delta \vdash a \# f \ t$}
        \DisplayProof
    &
\end{tabular}
\end{frame}

\begin{frame}{Derivation Rules for $\alpha$-Equivalence}
\scalebox{0.9}{
\begin{tabular}{c c}
    % unit 
        \AxiomC{}
        \RightLabel{($\aeq  \langle \rangle$)}
        \UnaryInfC{$\Delta \vdash \langle \rangle \aeq \langle \rangle$}
        \DisplayProof 
    &
    % atom
        \AxiomC{}
        \RightLabel{($\aeq  atom$)}
        \UnaryInfC{$\Delta \vdash a \aeq a$}
        \DisplayProof 
    \\ \\ 
    % application
        \AxiomC{$\Delta \vdash s \aeq t$}
        \RightLabel{($\aeq  app$)}
        \UnaryInfC{$\Delta \vdash f s \aeq f t$}
        \DisplayProof 
    &  
    % abstraction a
        \AxiomC{$\Delta \vdash s \aeq t$}
        \RightLabel{($\aeq  [a]a$)}
        \UnaryInfC{$\Delta \vdash [a]s \aeq [a]t$}
        \DisplayProof 
    \\ \\
    %% abstraction b
        \AxiomC{$\Delta \vdash$  $s \aeq (a \ b) \cdot t$, $a\#t$}
        \RightLabel{($\aeq [a]b$)}
        \UnaryInfC{$\Delta \vdash [a]s \aeq [b]t$}
        \DisplayProof 
    &
    %% variable
        \AxiomC{$ds(\pi, \pi')\#X \subseteq \Delta$}
        \RightLabel{($\aeq var$)}
        \UnaryInfC{$\Delta \vdash \pi \cdot X \aeq \pi' \cdot X$}
        \DisplayProof 
    % pair
    \\ \\ 
        \AxiomC{$\Delta \vdash s_0 \aeq t_0$, \ $\Delta \vdash s_1 \aeq t_1$}
        \RightLabel{($\aeq pair$)}
        \UnaryInfC{$\Delta \vdash \langle s_0, s_1 \rangle \aeq \langle t_0, t_1
        \rangle$}
        \DisplayProof 
    
\end{tabular}}
\end{frame}

\begin{frame}{Additional Rule for $\alpha$-Equivalence with Commutative Symbols}
We need to add a rule to take into account commutative function symbols. Therefore, if a
function symbol is commutative, the following rule can be applied:
\newline \newline
\begin{tabular}{c}
    % commutative pair
        \AxiomC{$\Delta \vdash s_0 \aeq t_1$, \ $\Delta \vdash s_1 \aeq t_0$}
        \RightLabel{($\aeq C-pair$)}
        \UnaryInfC{$\Delta \vdash f(\langle s_0, s_1 \rangle) \aeq f(\langle t_0, t_1
        \rangle)$}
        \DisplayProof 
\end{tabular} 
\end{frame}

\begin{frame}{Derivation Rules as a Sequent Calculus}
    The derivation rules for freshness and $\alpha$-equivalence are a sequent calculus. 

    Deriving $a\#\langle X, [a]Y\rangle$ with  $\Delta = \{a\#X\}$:
    \newline \newline \newline
    \begin{tabular}{c}
        \AxiomC{$a\#X$}
        \AxiomC{}
        \RightLabel{($\#[a]a$)}
        \UnaryInfC{$a\#[a]Y$}
        \RightLabel{(\#pair)}
        \BinaryInfC{$a\#\langle X, [a]Y\rangle$}
        \DisplayProof 
    \end{tabular} 
\end{frame}

\begin{frame}{Derivation Rules as a Sequent Calculus}
    Deriving $[a]a \aeq [b]b$:
    \newline \newline \newline
    \begin{tabular}{c}
        \AxiomC{}
        \RightLabel{($\aeq  atom$)}
        \UnaryInfC{$a \aeq (a \ b) \cdot b$}
        \AxiomC{}
        \RightLabel{($\# atom$)}
        \UnaryInfC{$a\# b$}
        \RightLabel{($\aeq [a]b$)}
        \BinaryInfC{$[a]a \aeq [b]b$}
        \DisplayProof 
    \end{tabular} 
\end{frame}

