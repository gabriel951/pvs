\section{Background}
\subsection{Nominal Terms, Permutations and Substitutions}
\begin{frame}{Atoms and Variables} 
    Consider a set of variables $\varSet = \{X, Y, Z, \dots \}$ and 
    a set of atoms $\atomSet = \{a, b, c, \dots \}$.
\end{frame}

% permutation
\begin{frame}{Permutations}
    A permutation $\pi$ represents an exchange of a finite amount of atoms in
    $\atomSet$
    and is usually represented by a list of swappings: 
\begin{equation*}
    \pi = (a_1, b_1):: ... ::(a_n, b_n)::nil 
\end{equation*}
\end{frame}

% definition of nominal terms
\begin{frame}{Nominal Terms}
    \begin{definition}[Nominal Terms]
    Nominal terms are inductively generated according to the grammar: 
    \begin{equation*}
        s,t \ \ ::= \ \ \langle \rangle \ \ | \ \ a \ \ | \ \ \pi \cdot X \ \ | \ \ [a]t \ \ |
                    \ \ \langle s, t \rangle \ \       | \ \ f \ t \ \   
    \end{equation*}
    \end{definition}
    The symbols denote respectively: unit, atom term, moderated variable,
    abstraction, pair and function application.
\end{frame}

% permutation action
\begin{frame}{Permutation Action on an Atom}
    \begin{definition}[Permutation Action on an Atom]
    The action of a permutation on an atom is defined by induction: 
    \begin{itemize} 
        \item $nil \cdot a \coloneqq a$ 
        \item $(b \ c):: \pi \cdot a \coloneqq \pi \cdot a$ 
        \item $(b \ c):: \pi \cdot b \coloneqq \pi \cdot c$ 
    \end{itemize}
    \end{definition}
\end{frame}

\begin{frame}{Permutation Action on a Term}
    \begin{definition}[Permutation Action on a Term]
    The action of a permutation $\pi$ on a term $t$ is defined by induction:
    \begin{align*}
        \pi \cdot  \langle \rangle &= \langle \rangle & 
        \pi \cdot (\pi' \cdot X) &= (\pi \oplus \pi') \cdot X & 
        \\ 
        \pi \cdot \langle s, t \rangle &= \langle \pi \cdot s, \pi \cdot t \rangle  & 
        \pi \cdot ([a]t) &= [\pi \cdot a](\pi \cdot t) & 
        \\
        \pi \cdot (f \ t) &= f(\pi \cdot t) 
    \end{align*}
    \end{definition}
\end{frame}

% substitution 
\begin{frame}{Substitution}
    \begin{definition}[Substitution]
    A substitution $\sigma$ is a mapping from variables to terms, such that
    $\{X \ | \ X \neq X\sigma \}$ is finite.
    \end{definition}
\end{frame}

% substitution action
\begin{frame}{Substitution Action on Terms}
    \begin{definition}[Substitution Action on a Term]    
        The action of a substitution $\sigma$ on a term $t$ is denoted by $\sigma t$ and
        defined recursively by: 
    \begin{align*}
        \langle \rangle \sigma &= \langle \rangle & 
        (\pi \cdot X) \sigma &= \pi \cdot (X \sigma)  & 
        \\ 
        \langle s, t \rangle \sigma &= \langle  s \sigma, t \sigma \rangle  & 
        ([a]t) \sigma &= [a](t \sigma) & 
        \\
        (f \ t) \sigma &= f (t \sigma) 
    \end{align*}
    
    OBS: Substitutions and permutations commute. 
    \end{definition}
\end{frame}

\subsection{Freshness and $\alpha$-Equality}

\begin{frame}{Intuition Behind the Concepts}
    Two important predicates are the freshness predicate $\#$ and the $\alpha$-equality
    predicate $\aeq$: 
    \begin{itemize}
        \item $a\#t$ intuitively means that if $a$ occurs in $t$ then it must do so
            under an abstractor $[a]$. 
        \item $s \aeq t$ intuitively means that $s$ and $t$ are $\alpha$-equivalent.
    \end{itemize}
\end{frame}

% TODO: put here what is a freshness context 
\begin{frame}{Contexts}
    TO DO. 
\end{frame}


\begin{frame}{Derivation Rules for Freshness}
    \begin{tabular}{ c c }
    % fresh empty tuple
        \AxiomC{}
        \RightLabel{($\# \langle \rangle$)}
        \UnaryInfC{$\Delta \vdash a \# \langle \rangle$}
        \DisplayProof
    &
    % fresh atom
        \AxiomC{}
        \RightLabel{($\# atom$)}
        \UnaryInfC{$\Delta \vdash a \# b$}
        \DisplayProof
    \\ \\  
    % perm - fresh
        \AxiomC{$(\pi^{-1}(a) \# X) \in \Delta $}
        \RightLabel{($\# X$)}
        \UnaryInfC{$\Delta \vdash a \# \pi \cdot X$}
        \DisplayProof
    &
    % abstraction - a 
        \AxiomC{}
        \RightLabel{($\# [a]a$)}
        \UnaryInfC{$\Delta \vdash a \# [a]t$}
        \DisplayProof
    \\ \\ 
    % abstraction - b 
        \AxiomC{$\Delta \vdash a \# t$}
        \RightLabel{($\# [a]b$)}
        \UnaryInfC{$\Delta \vdash a \# [b]t$}
        \DisplayProof
    & 
    % pair
        \AxiomC{$\Delta \vdash a \# s \ \ \ \Delta \vdash a \# t$}
        \RightLabel{($\# pair$)}
        \UnaryInfC{$\Delta \vdash a \# \langle s, t \rangle$}
        \DisplayProof
    \\ \\
    % application of function
        \AxiomC{$\Delta \vdash a \# t $}
        \RightLabel{($\# app$)}
        \UnaryInfC{$\Delta \vdash a \# f \ t$}
        \DisplayProof
    &
\end{tabular}
\end{frame}

\begin{frame}{Derivation Rules for $\alpha$-Equivalence}
    TO DO. 
\end{frame}

\begin{frame}{Additional Rule for $\alpha$-Equivalence with Commutative Symbols}
    TO DO. 
\end{frame}
