%TODO: Make a revision in notation commands, they are very confusing...

% General
\newcommand{\signature}{\Sigma}             %The signature set.
\newcommand{\subs}{\mathcal{S}ub}           %The substitutions set.
\newcommand{\vars}[1]{\mathtt{vars}(#1)}   %The set of variables.
\newcommand{\dom}[1]{\mathtt{dom}(#1)}     %The domain.
\newcommand{\ran}[1]{\mathtt{ran}(#1)}      %The range of a substitution.
\newcommand{\vran}[1]{\mathtt{v}\ran{#1}}

% Nominal Terms and Equality
\newcommand{\atomSet}{\mathbb{A}}           %The set of all atoms.
\newcommand{\varSet}{\mathbb{X}}               %The set of all META variables.
\newcommand{\abs}[2]{\left[ #1 \right]#2}   %Abstraction term.
\newcommand{\support}[1]{\mathtt{supp}(#1)}          %Support of an permutation.
\newcommand{\permID}{\mathtt{id}}           %Identity permutation.
\newcommand{\pAction}[2]{#1 \cdot #2}       %The action of a permutation on a term.
\newcommand{\pGroup}{\mathbb{P}}            %The group of all permutations acting on the set fo atoms.
\newcommand{\swapping}[2]{(#1 \ \ #2)}      %The Swapping permutation.
\newcommand{\terms}{T(\signature,\atomSet,\var)}    %In context of nominal this is the set of terms.

\newcommand{\fresh}{\#}
\newcommand{\instOrder}{\leq}
\newcommand{\aeq}{\mathrel{\approx_\alpha}}
\newcommand{\adis}{\mathrel{\not\approx_\alpha}}

\newcommand{\pair}[2]{\left\langle #1 \mid\mid #2 \right\rangle}

\newcommand{\pairEx}[3]{\pair{#1}{#2} - #3}

\newcommand{\dset}[2]{\mathtt{ds}(#1,#2)}

\newcommand{\pNF}[1]{\left\langle #1 \right\rangle\downarrow\,}


\newcommand{\algUnif}[1]{\mathtt{Unif}\left(#1\right)}

%\newcommand{\solution}[1]{\mathcal{S}(#1)} %solution set of the problem P
\newcommand{\eqUnif}{\overset{?}{\aeq}} %an unification equational problem
\newcommand{\eqDis}{\overset{?}{\adis}}
\newcommand{\freshPr}{\overset{?}{\fresh}}
\newcommand{\sol}[1]{\mathcal{U}\left(#1\right)}

