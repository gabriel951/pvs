\section{Introduction}

\begin{frame}{Nominal Syntax}
    Nominal syntax extends first order syntax bringing mechanisms to deal with bound
    and free variables, that is with $\alpha$-conversion in a natural manner.
    \par Profiting from the nominal paradigm implies adapting basic notions 
    (substitution, rewriting, equality, ...) to it.
\end{frame}

\begin{frame}{Purpose of Presentation}
    \par In this talk the problem of nominal unification with commutative operators
    is revisited, and a functional algorithm for nominal C-unification is presented. 
\end{frame}

% if i want to put more things, say that nominal rewriting is part of an effort to
% more general formalisms than lambda calculus or 1st order rewriting, as said in
% Maribels article. 
