\section{Introduction}

\begin{frame}{Term Rewriting Motivation}
    Term rewriting is a branch of (theoretical) computer science with applications in 
    algebra, software engineering and programming languages. 
\end{frame}

\begin{frame}{Nominal Rewriting Motivation}
\begin{itemize} 
    \item Nominal rewriting extends first order term rewriting, using nominal sets.
    \item This allows systems with bindings to be represented, for instance 
        $\lambda$-calculus $\beta$-reduction rules. 
\end{itemize}
\end{frame}

% if i want to put more things, say that nominal rewriting is part of an effort to
% more general formalisms than lambda calculus or 1st order rewriting, as said in
% Maribels article. 

\begin{frame}{Purpose of Presentation}
    In nominal rewriting, there is the concept of nominal unification. Nominal C
    unification expands this concept to 
    take into account commutativity. In this talk, nominal C unification is explained.  
\end{frame}


